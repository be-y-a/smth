% !TEX encoding = UTF-8 Unicode
\documentclass[12pt]{article}
\usepackage[utf8]{inputenc}
\usepackage[english, russian]{babel}
% !TEX encoding = UTF-8 Unicode
\usepackage{amssymb, amsmath, natbib, amsthm, graphicx, euscript, mathrsfs, enumerate, empheq}
\renewcommand{\t}[1]{\tau^{(#1)}}
\newcommand{\hgamma}{\hat{\gamma}_{n}}
\newcommand{\bgamma}{\bar{\gamma}_{n}}
\newcommand{\mG}{\mathcal{G}}
\newcommand{\wvn}{\widetilde\varepsilon_{n}}
\newcommand{\A}{\mathscr{A}}
\newcommand{\Par}{\mathscr{P}}
%\newcommand{\eqd}{\stackrel{\mathcal{L}}{=}}
\newcommand{\eqd}{\stackrel{d}{=}}
\newcommand{\lsim}{\lesssim }
\newcommand{\gsim}{\gtrsim }
\newcommand{\corr}{\operatorname{corr}}
\newcommand{\V}{\mathcal{V}}
\newcommand{\BG}{\texttt{BG}}
\newcommand{\PP}{{\mathbb P}}
\newcommand{\Q}{{\mathbb Q}}
\newcommand{\n}{\newline}
\newcommand{\tchi}{\tilde\chi}
\newcommand{\bchi}{\breve\chi}
\newcommand{\bpsi}{\breve\psi}
\newcommand{\bphi}{\breve\phi}
\newcommand{\sign}{\operatorname{sign}}
\newcommand{\diag}{\operatorname{diag}}
\newcommand{\Matr}{\operatorname{Matr}}
\newcommand{\T}{\mathcal{T}}
\newcommand{\E}{{\mathbb E}}
\newcommand{\D}{{\mathbb D}}
\newcommand{\LL}{{\EuScript L}}
\renewcommand{\i}{\mathrm{i}}
\newcommand{\hs}{\hat{s}}
\newcommand{\too}{\rightsquigarrow}
%\def\R{I\!\!R}
%\def\N{I\!\!N}
\newcommand{\Z}{{\mathbb{Z}}}
\newcommand{\N}{{\mathbb{N}}}
\newcommand{\R}{{\mathbb{R}}}
\newcommand{\CC}{{\mathbb{C}}}
\newcommand{\eps}{\varepsilon}
\newcommand{\Ree}{\operatorname{Re}}
\newcommand{\Var}{\operatorname{Var}}
\newcommand{\cov}{\operatorname{cov}}
\newcommand{\supp}{\operatorname{supp}}
\newcommand{\Exp}{\operatorname{Exp}}
\newcommand{\Pois}{\operatorname{Pois}}


\newtheorem{thm}{Теорема}[section]
\newtheorem{lem}[thm]{Лемма}
\newtheorem{cor}[thm]{Следствие}
\newtheorem{prop}[thm]{Утверждение}
\newtheorem{rem}[thm]{Замечание}

\theoremstyle{definition}
\newtheorem{defi}[thm]{Определение}
\newtheorem{ex}[thm]{Пример}

%\renewtheorem{def}{Defition}
%\DeclareMathOperator*{\argmax}{arg\,max}
%\DeclareMathOperator*{\argmin}{arg\,min}
\newcommand{\argmin}{\operatornamewithlimits{arg\,min}}
\newcommand{\argmax}{\operatornamewithlimits{arg\,max}}

\renewcommand{\P}{{\mathbb P}}
\renewcommand{\L}{\mathscr{L}}
\newcommand{\F}{\mathscr{F}}
\newcommand{\dw}{\widetilde{w}}
\newcommand{\bw}{\bar{w}}
\newcommand{\mY}{\mathcal{Y}}
\newcommand{\B}{\mathcal{B}}
\newcommand{\W}{\mathcal{W}}
\newcommand{\vX}{\vec{X}}
\newcommand{\vW}{\vec{W}}
\newcommand{\vu}{\vec{u}}
\newcommand{\vt}{\vec{t}}
\newcommand{\vmu}{\vec{\mu}}
\newcommand{\bd}{\bar{\delta}}
\newcommand{\wY}{\widetilde{Y}}
\renewcommand{\kappa}{\varkappa}
\def\sec#1{\underline{\textbf{#1}}}
\newcommand{\p}{\Upsilon}
\renewcommand{\k}{k}
\renewcommand{\wp}{\widetilde{\p}}
\def\g#1{\bar{g}_{#1}(J)}
\newcommand{\bz}{\breve{\zeta}}
\newcommand{\bp}{\breve{\p}}
\newcommand{\btau}{\breve{\tau}}
\newcommand{\cc}{\breve{c}}
\newcommand{\DD}{\mathcal{D}}
\newcommand{\I}{{\mathbb I}}
\newcommand{\NN}{\EuScript{N}}
\newcommand{\tN}{\widetilde{N}}

\newcommand{\tol}{\xrightarrow{Law}}
\newcommand{\toP}{\xrightarrow{\P}}
\newcommand{\lr}{\leftrightarrow}
\newcommand{\wK}{\gamma}%\widetilde{K}}
\newcommand{\pa}{\partial}

\newcommand{\mm}{s}
\newcommand{\MM}{S}

\begin{document}
\hspace{10cm}
%\today
\begin{center}
\section*{%Курс ``Теория случайных процессов''. 
Домашнее задание № 1.}
\end{center}
\textit{Тема: Оценивание параметров. Экспоненциальные семейства распределений }\newline \newline
\textit{Крайний срок сдачи: 11 октября 2020 г. (до конца дня).}\newline \newline

\textit{Домашнее задание состоит из пяти теоретических задач T1-T4, T5*, четырёх вычислительных заданий N1-N4.  Максимальный балл за  T1 - T4, N1 - N4  равен 1.25, а за бонусную задачу   T5* - 2 балла. Баллы выставляются с шагом 0.25 (то есть, можно получить 0.25 / 0.5 / ... баллов за одну задачу). Таким образом, оценка за домашнюю работу является (возможно, нецелым) числом от 0 до 12. \newline\newline
Итоговая оценка за курс вычисляется по формуле 
\begin{eqnarray*}
\min\Bigl( 
  \operatorname{median}(x_1, ..., x_n), 10 
\Bigr),
\end{eqnarray*}
где \(x_1,...,x_n\) - оценки за домашние работы.\newline
}


\textit{Решение нужно прислать через Ozon Masters bot  @ozonm\_bot в виде \textbf{одного PDF файла (в любом другом формате решения проверяться не будут}). Этот PDF файл должен содержать \newline
 - решения теоретических задач T1-T4, T5* набранные в LaTeX, Word,... или написанные от руки и затем отсканированные;\newline
 - программный код для численных заданий N1-N4;\newline
 - графики, показывающие, что код работает корректно. }
\newpage
\section{}
Два игрока играют в игру "камень-ножницы-бумага". Первый игрок выбрасывает камень, ножницы или бумагу с вероятностями \(p, q, (1-p-q)\) соответственно  (\(p,q \in [0,1]\)-константы, \(p+q\leq 1\)).

Второй игрок сначала сначала разыгрывает две i.i.d. величины \(\eta_1,\eta_2\) c распределением \(P\) на [0,1] и затем выбрасывает камень, ножницы или бумагу с вероятностями \(\eta_{(1)}, \eta_{(2)} - \eta_{(1)}, 1- \eta_{(2)}\) соответственно, где 
\begin{eqnarray*}
\eta_{(1)} = \min(\eta_1, \eta_2), \qquad 
\eta_{(2)} = \max(\eta_1, \eta_2).
\end{eqnarray*}

Предполагается, что перед началом игры было сыграно несколько раундов для разминки.


%\item Опишите статистическую модель, описывающую игру.
\begin{enumerate}
\item[T1] Вычислите значения параметров \(p\) и \(q\), которые максимизируют вероятность победы  первого игрока. Объясните, каким образом он может оценить эти значения на основе результатов разминочных раундов.
\item[N1] Просимулируйте игру для случаев, когда \(P\) является равномерным распределением на \([0,1]\) и бета распределением с параметрами \(\alpha=1, \beta=2\), а \(p_1, q_1\)  меняется от 0 до 1 с шагом 1/3. Для каждого набора распределения \(P\) и каждого набора \(p_1, q_1\) повторите игру \(100\) раз и сравните вероятности победы первого игрока. %Постройте диаграммы размаха для количества побед первого игрока в сериях длины 10 (то есть в играх под номерами 1-10, 11-20, ..., 91-100)
\end{enumerate}

\section{}
\begin{enumerate}
\item[T2]
Пусть \(X_1, ..., X_n\) - выборка из равномерного распределения на отрезке \([0,1]. \)  Обозначим   \(r-\)ую порядковую статистику через \(X_{(r)}\). Докажите, что 
\begin{enumerate}[(i)]
\item \(\E X_{(r)} = r/(n+1);\)
\item \(\E X^2_{(r)}=r(r+1)/((n+1)(n+2));\) 
\item мода (максимальное значение функции плотности) величины \(X_{(r)}\) равно \((r-1)/(n-1).\)
\end{enumerate}
\item[N2] Известная теорема об асимптотической нормальности выборочных квантилей гласит, что
\begin{eqnarray}\label{tt}
\sqrt{n} \Bigl( 
X_{(\lfloor \alpha n \rfloor +1)} - x_\alpha \Bigr) \tol \NN\Bigl(0, \frac{\alpha(1-\alpha)}{p^2(x_\alpha)}\Bigr), \qquad n \to \infty,
\end{eqnarray}
где \(\alpha \in (0,1),\) \(x_\alpha-\) теоретическая квантиль (то есть решение уравнения \(F(x)=\alpha\))), \(\NN(0,\cdot)\) - нормальное распределение со средним 0 и дисперсией \(\cdot\). Предполагается, что распределение является абсолютно непрерывным с плотностью \(p\), а \(\alpha\)  выбрано таким образом, что \(p(x_\alpha)>0\), см. [Лагутин М.Б. "Наглядная математическая статистика"\,, 2007, стр.88]. 

Пусть \(X\) имеет экспоненциальное распределение с функцией распределения
\begin{eqnarray*}
F(x) = 1 -  e^{-\lambda x}, \qquad x>0,
\end{eqnarray*}
с параметром \(\lambda>0.\)  Перед проведением численного эксперимента, описанного ниже, зафиксируйте параметры \(\alpha\) и \(\lambda.\)
\begin{enumerate}[(i)]
\item  Промоделируйте 100 выборок размера \(n=1000\) c этим распределением. 
\item Для каждой выборки, оцените левую часть~\eqref{tt}. 
\item Постройте график квантиль-квантиль, сравнивающие  эмпирические квантили в левой части~\eqref{tt} с теоретическими квантилями нормального распределения. 
\item Повторите шаги (i)-(iii) для \(n=10000, \) \(n=100000\). Убедитесь, что с увеличением \(n\) распределение приближается к нормальному. 
\item  Оцените дисперсию  выборок при каждом \(n\). Убедитесь, что дисперсия приближается к значению дисперсии предельного закона. 
\end{enumerate}
\section{}
\begin{enumerate}
\item[T3]
Дана выборка  из распределения Лапласа с плотностью распределения 
\begin{eqnarray*}
p(x, \theta)  = \frac{1}{2 \sigma} e^{-|x-\mu|/\sigma}, \qquad x \in \R.\end{eqnarray*}
где \(\mu \in \R, \sigma>0\) - параметры. Найдите оценки параметров \(\mu, \sigma\) 
\begin{enumerate}[(a)]
\item методом максимального правдоподобия; 
 \item методом моментов.
\end{enumerate}
\item[N3] Зафиксируйте значения \(\mu, \sigma\)  и просимулируйте случайную величину, имеющую распределение Лапласа. Повторите симуляции \(M=100\) раз, и по каждой выборке оцените параметры \(\mu\) и \(\sigma\) методом моментов и методом максимума правдоподобия. Постройте диаграммы размаха, показывающие, какой метод лучше.
\end{enumerate}
\section{}
\begin{enumerate}
\item[T4] Обозначим семейство распределений 
\begin{eqnarray*}
P_\theta= \Bigl\{ Law(\xi^2), \qquad
\xi \sim \NN(0,\theta) \Bigr\}.
\end{eqnarray*}
\begin{enumerate}[(a)]
\item Докажите, что данное семейство является экспоненциальным. 
 \item Используя только свойства экспоненциальных семейств:
 
- найдите математическое ожидание и дисперсию величины \(X\);

- предполагая, что дана выборка \(x_1,...,x_n\), найдите оценку параметра \(\theta\)  методом максимального правдоподобия и методом моментов.
\end{enumerate}
\item[N4]  Пусть \(X_0, X_1, X_2,...\) -  цены акций   в моменты времени \(0, 1,2,...\). Предположим, что теоретические лог-доходности
\begin{eqnarray*}
Y_k  = \log\bigl( X_k / X_{k-1} \bigr), \qquad k=1,2,...
\end{eqnarray*}
являются i.i.d. нормально распределёнными случайными величинами со средним 0 и неизвестной дисперсией \(\theta.\) Параметр \(\theta\) в этой модели  называют волатильностью цены.  \newline\newline
Рассмотрите цены акции некоторой компании (например,  IBM - data(ibm) в пакете waveslim) и разделите всю временную шкалу на 10 примерно одинаковых по длине временных интервалов. Для каждого интервала оцените волатильность цены акции. Визуально проверьте, что резкие изменения в цене приводят к резким изменениям волатильности.
\end{enumerate}
 \section{}
 \begin{enumerate}
\item [T5*] Пусть \(X_1, X_2,... X_n-\) последовательность i.i.d. случайных величин с равномерным распределением на отрезке \([a,b]\).  Используя понятие достаточной статистики, докажите, что набор величин 
\begin{eqnarray*}
Z_i = \frac{X_{(i)} - X_{(1)}}{X_{(n)}-X_{(1)}}, \qquad i=2,...,(n-1)
\end{eqnarray*}
и вектор \((X_{(1)}, X_{(n)})\) являются независимыми.
\end{enumerate}


\end{document} 